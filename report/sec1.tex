\section{Introduction}

Access to healthcare services is a fundamental element in ensuring the right to health and reducing territorial inequalities. 
Everywhere, the distribution of the population and the location of healthcare facilities significantly affect how easily citizens can obtain medical care.
Geographic distance, population density, and the availability of transport infrastructure play a crucial role in determining actual accessibility to essential services, thereby contributing to differences between urban, rural, and mountainous areas.

This project aims to analyze the ease of access to healthcare in Italy by integrating three datasets provided by Eurostat and the GISCO portal:

\begin{itemize}
	
	\item \emph{EU Population (Census 2021)}: contains information on the distribution of the European population in 1 km\textsuperscript{2} grids, useful for assessing demographic density and spatial distribution.

	\item \emph{Geographic Accessibility to Healthcare Services}: provides spatial data on distance and travel time to healthcare facilities, allowing the estimation of geographic accessibility to medical services.

	\item \emph{EU Communes (Municipalities)}: includes the administrative boundaries of European municipalities, essential for aggregating and comparing data at the local level.
\end{itemize}

By combining and analyzing these sources, it will be possible to build a detailed picture of the Italian situation, highlighting both the areas with the best access to healthcare and those facing greater challenges.

Starting from these data, the study wants to answer the following questions:
\begin{enumerate}
	\item Calculate the distribution of population in the country.
	\item Determine the mean accessibility value to healthcare services in each municipality.
	\item Determine the amount of population living in the municipalities where the mean accessibility level is low.
	\item Compare the mean accessibility level value in 2020 and 2023, so as to determine whether there has been an improvement or not.
\end{enumerate}

This report will be structured as follows:
\begin{itemize}
	\item \emph{Section 2}: description of used datasets.
	\item \emph{Section 3}: data preprocessing.
	\item \emph{Section 4}: demographic analysis.
\end{itemize}