\section{Dataset presentation}

For this project, three datasets provided by GISCO have been used.
GISCO, the \emph{Geographic Information System of the European Commission}, is the platform responsible for managing and disseminating geographical and cartographic data related to European statistics.
It is maintained by Eurostat, the statistical office of the European Union.

The three dataset chosen for this study are the following:
\begin{itemize}
	\item \emph{Census grid 2021}
	\item \emph{Geographic accessibility}
	\item \emph{Communes}
\end{itemize}


\subsection{Census grid 2021}

The Census Grid 2021 dataset, published by Eurostat, provides a harmonized representation of the resident population across the European Union based on the reference year 2021.
The dataset distributes population counts into a regular grid of 1 km\textsuperscript{2} cells, offering a highly detailed spatial perspective on demographic distribution.
Unlike administrative boundaries, which vary in size and shape across countries, the grid system ensures comparability and uniformity, making it particularly suitable for spatial analysis and cross-country studies.

Every cell contains the following demographic aggregated information:
\begin{itemize}
	\item Distribution by sex (male, female)
	\item Age group (under 15 years, 15-64, 65 years and over)
	\item Current activity status (number of employed persons)
	\item Place of birth (place of birth in reporting country, place of birth in other EU country, place of birth elsewhere)
	\item Place of usual residence (total population)
	\item Place of usual residence one year prior to the census (usual residence unchanged, move within the reporting country, move from outside the reporting country)
\end{itemize}

Dataset uses ETRS89 as a reference system.


\subsection{Geographic accessibility}
\label{subsec:geo_acc_datas_pres}

The Geographic Accessibility to Healthcare Services dataset provides harmonized information on the ease of reaching healthcare facilities across the European Union and EFTA countries. 
The dataset is based on a grid of cells of 1 km\textsuperscript{2} and contains indicators of accessibility expressed in terms of travel distance and travel time from each grid cell to the nearest healthcare service.
Travel time data rely on transport time measures using road network.

This dataset provides two indicators for accessibility:
\begin{itemize}
	\item driving time from each populated 1km\textsuperscript{2} cell to the nearest healthcare service
	\item the average driving time to the three nearest services.
\end{itemize}

Each time indicator is available for the reference years 2020 and 2023.

\subsection{Communes}

The Communes dataset contains the official administrative boundaries of municipalities (also referred to as communes or local administrative units, LAU level 2) across the European Union. 
Municipalities represent the lowest level of administrative governance in most European countries, making this dataset an essential reference for analyses that require a fine territorial granularity.

Here is the list of available fields:
\begin{itemize}
	\item COMM\_ID: identification code of the municipality;
	\item CNTR\_ID: here is simply the copy of CNTR\_CODE, but it can also be an internal identification number of Eurostat;
	\item CNTR\_CODE: country code as for ISO 3166-1 alpha-2;
	\item COMM\_NAME: name of the municipality;
	\item NAME\_ASCI: name of the municipality without special characters (ASCII characters);
	\item TRUE\_FLAG: a technical attribute used to distinguish real polygons from those that exist only for cartographic or topological reasons;
	\item NSI\_CODE: municipality code according to the National Statistics Institute;
	\item NAME\_NSI: name of the municipality according to the National Statistics Institute;
	\item NAME\_LATN: name of the municipality written with Latin characters. This is for countries which does not use Latin alphabet (Greece, Bulgaria, etc.);
	\item NUTS\_CODE: NUTS CODE (Nomenclature of territorial units for statistics).
\end{itemize}

Dataset uses ETRS89 as a reference system.









